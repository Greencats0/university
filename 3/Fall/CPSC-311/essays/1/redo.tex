\documentclass{article}
\usepackage{setspace}   %Allows double spacing with the \doublespacing command
\usepackage{lipsum} % Add dummy text

\RequirePackage{geometry}
\geometry{margin=1.0in}

\begin{document}

\doublespacing

\begin{center}
MEMORANDUM
\end{center}

\par\noindent\rule{\textwidth}{0.4pt}

\begin{flushleft}
TO: Pouya Radfar

FROM: Jared Dyreson

SUBJECT: Calgary Assignment

DATE: October 24, 2019
\end{flushleft}

\par\noindent\rule{\textwidth}{0.4pt}

Moving forward with the project of the assignment in Calgary as defined by scouting the location and reporting on costs for what that entails. 
In taking this position, I look forward to the additional responsibilities and new learning experiences.
With that being said, I wanted to bring to your attention to some concerns I do have found in regards to this assignment as stated below: 

\begin{enumerate}
\item The building needs to be refurbished, it is not up to code. Also it needs to be rewired for proper internet and phone lines.
\item There is still a need to hire employees for this location. Is there any word of hiring a recruiting manager or will this be handled directly through Human Resources?
\item For pay will it be competitive to compete with Company B as well? This goes for both new hires and transfers like myself.
\item Is there going to be a push to allow for more remote positions to incentivise working at the Calgary location?
\item What is the expected timeline of operations upon final completion?
\end{enumerate}

According to my calculations, it would cost around \$20,000 for removal of dial-up cables and rewiring of a building of this size.
Our new location has a size of 16,300 square feet and is centrally located in the heart of Calgary.
This office space has a dedicated parking garage that can support the estimated 250 people required to run this location.
Total number of spots is 350 which is an added buffer if we do in fact need to hire more individuals in the future.
When considering remote positions for our employees, it would allow for a wider pool of applicants as transportation would not be a driving factor.
There will be daily stand ups that each employee must participate in, describing what they are working on and it's estimated time of completion.
This ensures the employees are on the same schedule as the other engineers who are coming into the office.

As this project is in early stages of development, I have yet to make any new hires.
This process of hiring and creating job listings needs to be handled by an outside party.
I am requesting a branch of the current Human Resources Department to look into it or hire an outside entity to facilitate this.
I would work closely to this group to ensure the right listings are posted with very accurate and descriptive job listings

Since we are in direct competition with Company B, we need to offer competitive salaries.
The average salary for computer programmers, which is the field with most demand, is around \$55,000.
This means we need to increase our base rate for junior and mid-level programming positions to \$60,000 and offer weekly luncheons.

In regards to my salary, I am currently making \$45,000 which equates to 58826.03 Canadian. The average cost of living is for a four person household is 47916.00 Canadian.
This also does not take into account that the tax rate in Canada is 24\% which brings my annual salary to 44707.78 Canadian. 

With this current salary I cannot afford to maintain a post in Calgary. I would need at least a \$10,000 raise to meet the minimum threshold for supporting the cost of a family of four.

I want to make it clear that the projected time of completion should be done by early 2021, given the current circumstances.
This process can be expedited if the following can be achieved:

\begin{enumerate}
\item A separate entity to oversee the hiring process
\item My proper relocation to physically be present during the completion of this project
\item Ideas and thoughts regarding the competitive salary and incentives for working at this branch
\end{enumerate}

\par\noindent\rule{\textwidth}{0.4pt}

Sincerely,

Jared Dyreson

\end{document}
