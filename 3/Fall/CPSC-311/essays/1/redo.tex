\documentclass{article}
\usepackage{setspace}   %Allows double spacing with the \doublespacing command
\usepackage{lipsum} % Add dummy text

\RequirePackage{geometry}
\geometry{margin=1.0in}

\begin{document}

\doublespacing

\begin{center}
MEMORANDUM : SCENARIO 1
\end{center}

\par\noindent\rule{\textwidth}{0.4pt}

\begin{flushleft}
TO: Pouya Radfar

FROM: Jared Dyreson

SUBJECT: Calgary Assignment

DATE: October 24, 2019
\end{flushleft}

\par\noindent\rule{\textwidth}{0.4pt}

I am looking forward to the additional responsibility and new learning experiences with opening the new Calgary location. As discussed in our meeting two days ago, I would like to reiterate expectations. These include but are not limited to scouting the location and reporting on costs for what that entails. 
During my project planning, the following concerns/obstacles have arisen: 

\begin{enumerate}
\item The location requires additional infrastructure upgrades. Also it needs to be rewired for proper internet and phone lines. Given these hurdles, what is the company's expected timeline operations upon final completion?
\item There is still a need to hire employees for this location. Is there any word of hiring a recruiting manager or will this be handled directly through Human Resources?
\item Will salary be competitive to compete with the local economy and to compete with Company B as well? This goes for both new hires and transfers like myself.
\item Is there going to be a push to allow for more remote positions to incentivise working at the Calgary location?

\end{enumerate}

According to my calculations, it would cost around \$20,000 for removal of dial-up cables and rewiring of a building of this size.
Our new location has a size of 16,300 square feet and is centrally located in the heart of Calgary.
This office space has a dedicated parking garage that can support the estimated 250 people required to run this location.
Total number of spots is 350 which is an added buffer if we do in fact need to hire more individuals in the future.
Remote positions are employees who do not report to the physical Calgary location but instead work from home. This would allow for a wider pool of applicants as transportation would not be a deciding factor.
There will be daily stand ups that each employee must participate in, describing what they are working on and it's estimated time of completion.
This ensures the employees are on the same schedule as the other engineers who are coming into the office.

As this project is in early stages of development, I have yet to make any new hires.
This process of hiring and creating job listings needs to be handled by an outside party.
I am requesting a branch of the current Human Resources Department to look into it or hire an outside entity to facilitate this.
I would work closely to this group to ensure the right information is posted with very accurate and descriptive job listings.
A subset of these individuals would handle the actual interview and hiring process.

Since we are in direct competition with Company B, we need to offer competitive salaries.
The average salary for computer programmers, which is the field with most demand, is around \$55,000.
This means we need to increase our base rate for junior and mid-level programming positions to \$60,000 and offer weekly luncheons.
These incentives will attract more candidates because Company B is only offering the average salary and no other added bonuses.
Since our base of operations are in Canada, we do not have to worry about offering benefits to our employees.
That is handled through their taxes and employees have access to universal health care.

In regards to my salary, I am currently making \$85,000 which equates to 111014.25 Canadian.
This also does not take into account that the tax rate in Canada is 24\% which brings my annual salary to 84370.83 Canadian.
With my current salary, taking this post in Calgary would represent a net loss for me. I am open to discussing the total incentive package for a project of this scope.
I would also like to bring attention to my housing situation.
This would require me to sell my current house and purchase a new home.
Before the project goes into full effect, this needs to be addressed.
Also, since I am not a native Canadian citizen, I would either need to apply for a work visa or citizenship if this project is to be permanent.
Is it expected that this post is just temporary or does it require permanent residence in Canada for the foreseeable future?
With that being said, I also do not have access to the benefits of universal health care Canada provides.
Does my health care provided by our company translate identically and it does not matter which country I work in?
If it does matter, does that mean I need to find my own health care provider or does the company handle this?

I want to make it clear that the projected time of completion should be done by early 2021, given the current circumstances.
This process can be expedited if the following can be achieved:

\begin{enumerate}
\item A separate entity to oversee the hiring process, preferably with a local assistance to attract local/native workers to capture the culture of Calgary.
\item My proper immediate relocation to physically be present during the completion of this project
\item Ideas and thoughts regarding the competitive salary and incentives for working at this branch
\end{enumerate}

For any further questions regarding the status of this project, please call me to discuss.

\end{document}
